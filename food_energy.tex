% formatted for IOP
\documentclass[12pt]{iopart}

\pdfminorversion=4

\usepackage{float}
\usepackage{units}
\usepackage{graphicx}
\usepackage{hyperref}

\newcommand{\be}{\begin{equation}}
\newcommand{\ee}{\end{equation}}
\newcommand{\bea}{\begin{eqnarray}}
\newcommand{\eea}{\end{eqnarray}}
\newcommand{\degC}{^{\circ}C}
\begin{document}
\title[How many acres of potatoes does a society need?]{How many acres of potatoes does a society need?}
\author{N T Moore}
\footnote{Present address:
Department of Physics, Winona State University, Winona, MN 55987, USA}
\ead{nmoore@winona.edu}
\date{\today}
\begin{abstract}
One of the main difficulties in a class on Sources of Energy and Social Policy is the wide variety of units used by different technologists (BTU's, Barrels of oil, Quads, kWh, etc).  As every student eats, I think some of this confusion can be resolved by starting and grounding the class with a discussion of food and food production.  A general outline for this introduction is provided and two interesting historical cultural examples, Tenichtitlan and the Irish Potato Famine, are provided.  
Science and Social Policy classes are full of bespoke units and involve many different contexts.  Starting the class with a discussion of food energy is a nice way for everyone to start with the same context.  In addition, discussion of Food Energy can lead to interesting historical claims. 
\end{abstract}
\noindent{\it Keywords\/}: Energy, Social Policy, kcals, Tenochtitlan, Irish Potato Famine, History, self-reliance
\submitto{\PED}
\maketitle

\section{Introduction}
When the United States entered World War One one of the problems they faced was logistics.  How much food do you need to ship overseas to Europe to feed a million soldiers?  That early work in nutrition led to the 3000 Calorie diet many people remember from secondary Health Education class.  A bit about units you might remeber: $1~Calorie = 1~kilo-calorie~(kcal)$, and a dietician might build a $3000 kcal$ diet for a 20 year old basketball player.  A \textit{calorie} is the amount of energy it takes  to heat a gram of water by a degree Celsius.  There are about 4.2 Joules in a single calorie, and a Joule occurs all over introductory  physics.  If you need to buy a new home furnace, the sales brochure might advertise that it is capable of delivering 100,000 BTU's of heat each hour.  What's a BTU? Heat a pound of water by $1^{\circ}F$.  Of course Heat Pumps are far more efficient than simply burning methane or propane, but they consume kilo-watt-hours (kWh) of electricity, not BTU's.  What's a kWh?  Run a 1000 Watt toaster for an hour and you'll have pulled one kWh off the grid, it will cost you about \$0.13 in Minnesota.  If you decide to put solar panels in your backyard, they will probably collect about $10\%$ of the 3.5kWh the  the sun delivers to each square meter of your lawn (in Minnesota) each day.  

As the previous paragraph illustrates, there are a frustratingly large number of different units in an ``Energy'' class.  At Winona State, this 3 credit class fulfulls a ``Science and Social Policy'' general education requirement and is taken by students from across the university.   Lots of college majors don't require a math class beyond algebra or introductory statistics and the population is largely math-averse. You could jokingly say that one of the main things students learn in the class is unit converstion, but it isn't far off.  Nearly every field finds energy a useful representation, and every profession has their own set of units and terminology that's most well suited for quick calculation.  Would a medical lab scientist talk about the fractional acre-foot of urine needed test kidney function?  No, but someone in the central valley of California would certianly care about the acre-feet of water necessary to grow almonds!  Does a gas station price their gasoline in dollars per kWh? Given the growing electrification of cars, they might soon.

Everyone eats, maybe not $3000 kcals$ per day, but at least something every day.  When I teach our energy class, I spend a few weeks talking about food energy before all other types.  While food production is not central to climate change and wars over oil, food is essential in a way that diesel and gasoline are not.  Vehicle fuel makes modern life possible, but we could live, unpleasantly, without it.  We can't live without fats and protein.  

\section{Food Energy}

To introduce Food Energy, I ask the students to work through a few questions:

\subsection{Converting food into body heat}
Planning to save money, one college student decides to go to an all-you-can-eat buffet each day at 11am.  If he brings homework and stretches the meal out for a few hours he can get all $3000~kcals$ with only one meal bill.  Food is fuel for the human body -- could too much fuel make his body feel sick? If his body burned all this food at once, how much warmer would he get? 
Useful information: the student has a mass of 80kg and is made mostly of water.  A Calorie heats 1 kg of water $1^{\circ}C$. 

Here's a possible answer:
equate food energy with calorimetric heating and assume human bodies have the same heat capacity as water, about $1\frac{kcal}{kg\cdot\degC}$. This allows us to calculate the body's temperature increase.
\bea
3000kcals &=& 80kg\cdot1 \frac{kcal}{kg\cdot \degC}\cdot\Delta T\\
\Delta T &\approx& +37.5\degC
\eea
Students are normally quite surprised at this number.  Although wildly unrealistic, $\Delta T \approx +6\degC$ is typicaly fatal, there is a related phenomena of diet-induced thermogenesis\cite{meat_sweats} known informally as ``the meat sweats''. Some students connect this calculation to feeling quite hungry after a cold swim in the pool (a similar effect).  On a larger scale, discussing what's wrong with this estimate is useful.  The main storage mechanism for storing food energy is fat tissue, which the calculation completely ignores.  Infants are generally born with little fat, and an infant sleeping through the night often coincides with the baby growing enough fat tissue to store sufficient kcals to make it though a night without waking up ravenously hungry.  A related follow-up is that if a person is stranded in the wilderness, they should immediately start walking downstream (ie, towards civilization) as they likely won't be able to harvest an amount of kcals equivalent to what they already have stored on their hips and abdomen.\cite{trout}  The contrast of bear hibernation \cite{fat_bear} and songbirds constatly eating through the winter are related connections to investigate.

\subsection{Biophysical Power}
A more realistic question to follow up with relates to the average \textit{power} given off by a person over a day.  
Again, assuming $3000kcal$ is burned over $24 hours$, with useful information: $1 kcal \approx 4200J$ and $1 J/s=1W$.
\be
\frac{3000kcal}{24hours}\cdot\frac{4200J}{1kcal}\cdot\frac{1hour}{3600sec}\approx145W
\ee
Most students still remember $75Watt$ lightbulbs, but given the spread of LED lighting, ``A person's body heat is two 75W light bulbs'' will probably only make sense for a few more years.  Desert or cold-weather camping, alone versus with friends, and survival swimming are also examples for students to make sense of this answer.  If you can take advantage of other people's waste body heat, you'll sleep more pleasantly and survive longer in cold water.  

Another application to discuss is that of ``brown fat,'' a sort of biological space heater that humans and other mammals develop in response to cold weather.  This tissue's mitochondria can burn lipids and carbohydrates in a useless proton pumping scheme, which produces metabolic heat \cite{brown_fat}.  Most common in rodents and infants, this mechanism can be stimulated by extended exposure to cold temperatures.  The idea of a biological space heater that takes a month to turn on and a month to turn off matches the lived experience of college students in Minnesota, who wear down jackets in $4\degC$ weather in November, and beachwear in the same $4\degC$ weather in March.  Additionally, transplants to northern climates often take a few years to ``get used to'' the colder weather up north. It seams just as easy to say that transplants' bodies take a few years to develop the brown fat cells which allow them to be comfortable in cold weather.

One other distinction to emphasize is the difference between power and energy.  A graph of a human body's ``kcal content'' over the course of a day can be a useful illustration.  When sedentary, this graph probably has the slope of $-150W\approx -125 \frac{kcals}{hour}$.  If the $3000kcal$ meal at the buffet takes an hour, this period corresponds to an energy-time slope of $+3000\frac{kcal}{hour}\approx +3500W$.  

In medicine, these slopes are effectively equivalent to ``Metabolic Equivalent of Task'' (METS), a common measure in cardiology and exercise physiology.  METS is power normalized by mass, $1METS=1\frac{kcal}{kg\cdot hour}$, and METS levels are available for many different physcial activities. \cite{METS}

\subsection{Burning off food energy}
Imagine that after eating a $600~kcal$ bacon-maple long-john (donut), you decide to go for a hike to ``work off'' the Calories.  Winona State  is in a river valley bounded by $200m$ tall bluffs.  How high up the bluff would you have to hike to burn off the donut?  
Useful information: human muscle is about $1/3$ efficient, and on Earth's surface, gravitational energy has a slope of about $10~\frac{Joules}{kg\cdot m}$.

\begin{figure}[h]
\centering
\includegraphics[width=\columnwidth]{bar_chart.png}
\caption{An Energy Bar Chart to illustrate the $1/3$ efficient student hiking up a bluff to burn off the morning's donut.  The initial state (left) is the hiker at the bottom of the hill, with donut in stomach.  The final state (right) is the hiker at the top of the bluff with $2/3$ of the energy removed to the atmosphere by sweat and exhalation of warm air. $1/3$ of the donut's energy is stored in elevation.  The system for this diagram includes the earth, the hiker, and the donut.  The system does not include the atmosphere around the hiker.  
}
\label{bar_chart}
\end{figure}

One way to approach this problem is by using Energy Bar Charts \cite{energy_bar_charts} to illustrate how the energy held in food changes form as it is used.  An approximation for this question is shown in figure \ref{bar_chart}.  
In this story, the ``system'' is taken to be the earth, food, and hiker.  The hiker's body is assumed to be $1/3$ efficient, which means one of the food energy blocks of energy is transformed into gravitational energy (elevation) at the end of the hike.  
The other $2$ blocks of energy are transformed into heat and leave the hiker's body, most likely by mechanisms of respiration and sweat evaporation. The purpose of a bar chart like this is to provide a pictoral and mathematical representation of the energy conservation equation given in \ref{eq:bar_chart}.         

\bea
\frac{1}{3}\cdot600kcal\cdot\frac{4200J}{1kcal} 
	&=& 80kg\cdot10\frac{Joules}{kg\cdot m}\cdot height \label{eq:bar_chart}\\
height &\approx&  1000 m
\eea
This estimate is again surprising to students.  Five trips up the bluff to burn off $\$2$ of saturated fat, sugar, and flour!  A nice followup calculation is to imagine a car that can burn a $100kcal$ piece of toast in the engine: from rest, what speed will the toast propel it to? If (again) the engine converts $1/3$ of the energy into motion (kinetic energy), a 1300kg Honda Civic will reach a speed of about $13\frac{m}{s}\approx33mph$!  

The point of these energy conversion calculations is not to give students an eating disorder.  Rather, the numbers show food's amazing power. A single slice of toast will bring a car up to the residential speed limit!  A day's food, $3000kcal$, will power you up an $8000m$ mountain peak! The body-work food allows us to do is astonishing, and increases in food production have made modern  comforts, unimaginable 150 years ago, possible to the point of being taken for granted.  

\subsection{Where does food energy come from?}
One feature of the aught's ``homesteading'' culture \cite{homesteading} is the idea that a person should probably be able to move to the country, eat a lot of peaches, and grow all their own food.  Learning that farming labor is \textit{skilled} labor can be a brutal and disheartening realization. Eating $3000kcals$ each day means planting, weeding, harvesting, and storing more than a million kcals each year \cite{Haspel}.  Where will those Calories come from? Is your backyard enough to homestead in the suburbs \cite{backyard_homestead}?

At some point bewteen 1920 and 1950, US chemical manufacturers realized that in the post-war period, they could repurpose processes developed for manufacturing munitions and chemical warfare agents to produce chemicals that would kill insects and increase the nitrogen levels in the soil. 
As figure \ref{ag_yields} shows, the epoch of ``Better Living Through Chemistry'' produced a dramatic increase in per-acre yields across all comodity food crops.  

\begin{figure}[ht!]
\centering
\includegraphics[width=\columnwidth]{yield_over_time.png}
\caption{
USDA yields over time
The data plotted comes from the USDA
\url{https://www.nass.usda.gov/Statistics_by_Subject/index.php}
The idea for this plot came from an online blog, \cite{math_encounters}.  
Details for recreating this plot are given in \ref{how_yield_plot_is_made}.
}
\label{ag_yields}
\end{figure}

If you're discussing backyard Calorie production it isn't reasonable to use modern yield estimates for planning.  ``Roundup Ready'' Corn, Soybean, and Sugar Beet seeds are not available to the public, and the edge effects from deer and insects are much smaller on a $600$ acre field than they are in an community garden allotment.  As mentioned in the introduction, in 1917 the USDA published a pamphlet, shown in \ref{1917_yields} about yields a farmer might expect.  Their data came from pre-war, pre-chemical agriculture, and the yields cited were produced with horses, manure, lime, and large families full of children.  If you want to be self sufficient, these yield numbers are probably a good upper bound on what's realistically possible.  

\begin{figure}[ht!]
\centering
\includegraphics[width=\columnwidth]{USDA_1917_cropped.pdf}
\caption{
USDA yields from pre-chemical US ag
}
\label{1917_yields}
\end{figure}

So, another question using this data.  If you want to feed your family of 4 potatoes, how much land will you need to cultivate?
1917 data

Grow your own food, possible?  

Grow your own food, how far apart (urban life?)


\section{Example: How big could Tenochtitlan have been?}

1917 (A\&M) USDA pamphlet

Corn for US - area

If Tenoch was 100k people, how much land area?

\section{Example: Was the Irish Potato Famine a Natural Disaster?}

\section{Conclusion}


%\begin{acknowledgments}
\ack
The work was prompted in part by discussions with John Deming, Carl Ferkinhoff, and Sarah Taber.
%\end{acknowledgments}

%The command \appendix is used to signify the start of the appendices. Thereafter
%\section, \subsection, etc, will give headings appropriate for an appendix. To obtain
%a simple heading of ‘Appendix’ use the code \section*{Appendix}. If it contains
%numbered equations, figures or tables the command \appendix should precede it and
%\setcounter{section}{1} must follow it
\clearpage
\appendix
\section{Creating the historical kcal/acre figure from USDA data}
\label{how_yield_plot_is_made}
The United States Department of Agriculture (USDA) provides historical crop information via the National Agricultureal Statistics Service, \url{https://www.nass.usda.gov/Statistics_by_Subject/index.php?sector=CROPS}.  Data was downloaded in spreadsheet csv format and then combined and plotted via a Python Jupyter notebook.   

Each crop has its own bespoke units, for example potatoes are sold by hundredweight (CWT) but sugar beets are measured by the ton.  
Every imaginable agricultural product seems to be tracked in the NASS site, for example Maple Syrup production is tracked and given in gallons of syrup per (tree) tap! 
Conversion factors used are summarized in Table \ref{conversions}.  
Calorie (kcal) density for each crop was taken from \url{https://fdc.nal.usda.gov/fdc-app.html}.  Within this database, foods are identified by an FDC ID.  

An example calculation (implemented in the Jupyter notebook) follows for Corn.  
In 2022 the USDA reported an average production of 172.3 bushels of corn per acre of farmland.  
\be
172.3\frac{bu}{acre}\cdot\frac{56lbs~corn}{bu}\cdot\frac{453.592~grams}{lbs}\cdot\frac{365~kcal}{100~grams} = 15,974,657 \frac{kcal}{acre}
\ee
Obviously the result is only reasonable to two signifigant figures!
%grams_per_lbs=453.592
%corn_lbs_per_bu=(56.0/1.0)
%corn_kcal_per_gram=(365/100)
%corn_kcal_per_acre = corn_bu_per_acre*corn_lbs_per_bu*grams_per_lbs*corn_kcal_per_gram

\begin{table}
\caption{\label{label}
A summary of units and conversions used to create figure \ref{ag_yields} from USDA NASS data.  $1cwt$ is a hundred pounds of potatoes.  A bushel, $1bu$, is a volume unit of about 35liters and corresponds to about 60lbs of grain. Calorie content per 100 gram mass of food is taken from the USDA's ``Food Data Central'' database. It isn't clear from any of these resources if lb is pound-force (lbf) or pound-mass (lbm) and so I am ignorantly treating them as ``grocery store units'' where $1 lbs = 453.592 grams$.
}
\begin{indented}
\item[]\begin{tabular}{@{}lllll}
\br
Crop&per acre unit&production unit&kcals per 100gram &  FDC ID\\
\mr
Corn & bu/acre & $1bu=56lbs$ & 365 &170288 \\
Potatoes & cwt/acre & $1CWT=100lbs$ & 77 & 170026 \\
Soybeans & bu/acre & $1bu=60lbs$ & 446 & 174270 \\
Sunflowers & lbs/acre & & 584 & 170562 \\
Wheat & bu/acre & $1bu=60lbs$ & 327 & 168890 \\
\br
\end{tabular}
\end{indented}
\label{conversions}
\end{table}



\section*{References}
\begin{thebibliography}{99}

\bibitem{meat_sweats}
https://www.bonappetit.com/story/meat-sweats
P. Trayhurn,
THERMOGENESIS,
Editor(s): Benjamin Caballero,
Encyclopedia of Food Sciences and Nutrition (Second Edition),
Academic Press,
2003,
Pages 5762-5767,
ISBN 9780122270550,
https://doi.org/10.1016/B0-12-227055-X/01188-3.
(https://www.sciencedirect.com/science/article/pii/B012227055X011883)

\bibitem{trout}
The wilderness river might be full of trout, but if they're $300kcals$ each, you'll have to catch, clean, and smoke $10$ of them to store up a day's food. \url{https://fdc.nal.usda.gov/fdc-app.html#/food-details/175154/nutrients}

\bibitem{fat_bear}
Some sources claim that bear metabolism can vary between $4,000$ to $20,000$ kcals per day, \url{https://bear.org/5-stages-of-activity-and-hibernation/}, comically illustrated by the National Park Service at \url{https://www.nps.gov/katm/learn/fat-bear-week-2022.htm} .
  
\bibitem{brown_fat}
Huttunen P, Hirvonen J, Kinnula V. The occurrence of brown adipose tissue in outdoor workers. Eur J Appl Physiol Occup Physiol. 1981;46(4):339-45. doi: 10.1007/BF00422121. PMID: 6266825.

Brown and Beige Fat: Molecular Parts of a Thermogenic Machine
Paul Cohen1 and Bruce M. Spiegelman2
Diabetes 2015;64:2346–2351 | DOI: 10.2337/db15-0318

https://pubmed.ncbi.nlm.nih.gov/33846638/
Shamsi F, Piper M, Ho LL, Huang TL, Gupta A, Streets A, Lynes MD, Tseng YH. Vascular smooth muscle-derived Trpv1+ progenitors are a source of cold-induced thermogenic adipocytes. Nat Metab. 2021 Apr;3(4):485-495. doi: 10.1038/s42255-021-00373-z. Epub 2021 Apr 12. PMID: 33846638; PMCID: PMC8076094.

https://pubmed.ncbi.nlm.nih.gov/14715917/
Cannon B, Nedergaard J. Brown adipose tissue: function and physiological significance. Physiol Rev. 2004 Jan;84(1):277-359. doi: 10.1152/physrev.00015.2003. PMID: 14715917.

https://pubmed.ncbi.nlm.nih.gov/6722594/
Himms-Hagen J. Nonshivering thermogenesis. Brain Res Bull. 1984 Feb;12(2):151-60. doi: 10.1016/0361-9230(84)90183-7. PMID: 6722594.

https://www.nih.gov/news-events/nih-research-matters/uncovering-origins-brown-fat

\bibitem{METS}
Jetté M, Sidney K, Blümchen G. Metabolic equivalents (METS) in exercise testing, exercise prescription, and evaluation of functional capacity. Clin Cardiol. 1990 Aug;13(8):555-65. doi: 10.1002/clc.4960130809. PMID: 2204507.

\bibitem{energy_bar_charts}
something from AMTA?

Energy as a substancelike quantity that flows: Theoretical considerations
and pedagogical consequences
Eric Brewe
PHYSICAL REVIEW SPECIAL TOPICS - PHYSICS EDUCATION RESEARCH 7, 020106 (2011)
https://journals.aps.org/prper/abstract/10.1103/PhysRevSTPER.7.020106

\bibitem{Haspel}
In defense of corn, the world’s most important food crop
The Washington Post
Tamar Haspel 
July 12 2015


\bibitem{homesteading}
See for example, the Discover television show, ``Alaska the Last Frontier,'' any issue of ``Mother Earth News,'' or Backyard Chicken feeds on Instagram.  

\bibitem{backyard_homestead}
The Backyard Homestead: Produce all the food you need on just a quarter acre!
Carleen Madigan
Storey Publishing, LLC; 14th Printing edition (February 11, 2009)

\bibitem{math_encounters}
https://www.mathscinotes.com/2017/01/calorie-per-acre-improvements-in-staple-crops-over-time/
Mark Biegert 
Math Encounters 
2017

\bibitem{Aztec_Cannibalism} for crop productivity

\bibitem{USDA_1917_yields}

\end{thebibliography}
\end{document}
