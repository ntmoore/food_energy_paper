% formatted for IOP
\documentclass[12pt]{iopart}

\pdfminorversion=4

\usepackage{float}
\usepackage{units}
\usepackage{graphicx}
\usepackage{hyperref}

\newcommand{\be}{\begin{equation}}
\newcommand{\ee}{\end{equation}}
\newcommand{\bea}{\begin{eqnarray}}
\newcommand{\eea}{\end{eqnarray}}
\begin{document}
\title[How many acres of potatoes does a society need?]{How many acres of potatoes does a society need?}
\author{N T Moore}
\footnote{Present address:
Department of Physics, Winona State University, Winona, MN 55987, USA}
\ead{nmoore@winona.edu}
\date{\today}
\begin{abstract}
One of the main difficulties in a class on Sources of Energy and Social Policy is the wide variety of units used by different technologists (BTU's, Barrels of oil, Quads, kWh, etc).  As every student eats, I think some of this confusion can be resolved by starting and grounding the class with a discussion of food and food production.  A general outline for this introduction is provided and two interesting historical cultural examples, Tenichtitlan and the Irish Potato Famine, are provided.  
Science and Social Policy classes are full of bespoke units and involve many different contexts.  Starting the class with a discussion of food energy is a nice way for everyone to start with the same context.  In addition, discussion of Food Energy can lead to interesting historical claims. 
\end{abstract}
\noindent{\it Keywords\/}: Energy, Social Policy, kcals, Tenochtitlan, Irish Potato Famine, History, self-reliance
\submitto{\PED}
\maketitle

\section{Introduction}
When the United States entered World War One one of the problems they faced was logistics.  How much food do you need to ship overseas to Europe to feed a million soldiers?  That early work in nutrition led to the 3000 Calorie diet many people remember from secondary Health Education class.  A bit about units you might remeber: $1~Calorie = 1~kilo-calorie~(kcal)$, and a dietician might build a 3000 kcal diet for a 20 year old basketball player.  A \textit{calorie} is the amount of energy it takes  to heat a gram of water by a degree Celsius.  There are about 4.2 Joules in a single calorie, and a Joule occurs all over introductory  physics.  If you need to buy a new home furnace, the sales brochure might advertise that it is capable of delivering 100,000 BTU's of heat each hour.  What's a BTU? Heat a pound of water by $1^{\circ}F$.  Of course Heat Pumps are far more efficient than simply burning methane or propane, but they consume kilo-watt-hours (kWh) of electricity, not BTU's.  What's a kWh?  Run a 1000 Watt toaster for an hour and you'll have pulled one kWh off the grid, it will cost you about \$0.13 in Minnesota.  If you decide to put solar panels in your backyard, they will probably collect about $10\%$ of the 3.5kWh the  the sun delivers to each square meter of your lawn (in Minnesota) each day.  

As the previous paragraph illustrates, there are a frustratingly large number of different units in an ``Energy'' class.  At Winona State, this 3 credit class fulfulls a ``Science and Social Policy'' general education requirement and is taken by students from across the university.   Lots of college majors don't require a math class beyond algebra or introductory statistics and the population is largely math-averse. You could jokingly say that one of the main things students learn in the class is unit converstion, but it isn't far off.  Nearly every field finds energy a useful representation, and every profession has their own set of units and terminology that's most well suited for quick calculation.  Would a medical lab scientist talk about the fractional acre-foot of urine needed test kidney function?  No, but someone in the central valley of California would certianly care about the acre-feet of water necessary to grow almonds!  Does a gas station price their gasoline in dollars per kWh? Given the growing electrification of cars, they might soon.

Everyone eats, maybe not 3000 kcals per day, but at least something every day.  When I teach our energy class, I spend a few weeks talking about food energy before all other types.  While food production is not central to climate change and wars over oil, food is essential in a way that diesel and gasoline are not.  Vehicle fuel makes modern life possible, but we could live, unpleasantly, without it.  We can't live without fats and protein.  

\section{Food Energy}

To introduce Food Energy, I ask the students to work through a few questions:

Planning to save money, one college student decides to go to an all-you-can-eat buffet each day at 11am.  If he brings homework and stretches the meal out for a few hours he can get all $3000~kcals$ with only one meal bill.  Food is fuel for the human body. If his body burned all this food at once, how much warmer would he get? 
Useful information: the student has a mass of 80kg and is made mostly of water.  A Calorie heats 1 kg of water $1^{\circ}C$. 

Answer
\bea
3000kcals &=& 80kg\cdot1 \frac{kcal}{kg\cdot C^{\circ}}\cdot\Delta T\\
\Delta T &\approx& +37.5^{\circ}
\eea
Fat tissue serves a valuable purpose, brown fat, babies, songbirds

What power does the body give off in the more realistic case that the 3000kcal is burned over 24 hours? 
Useful information: $1 kcal \approx 4200J$ and $1 J/s=1W$.
\be
\frac{3000kcal}{24hours}\frac{4200J}{1kcal}\frac{1hour}{3600sec}\approx145W
\ee
Survival swimming, putting all the kids in one bed on a cold winter night.

Imagine that after eating a $600~kcal$ bacon maple long-john (donut), you decide to go for a hike to work off the Calories.  Winona State  is in a river valley bounded by 200m tall bluffs.  How high up the bluff would you have to hike to burn off the donut?  
Useful information: human muscle is about $30\%$ efficient and gravitational energy on Earth's surface has a slope of about $10~Joules/kg\cdot m$.

Answer
Energy bar charts
\bea
\frac{1}{3}\cdot600kcal\cdot\frac{4200J}{1kcal} 
	&=& 80kg\cdot10\frac{Joules}{kg\cdot m}\cdot height\\
height &\approx&  1000 m
\eea

increase in yields since 1917 (graph)

1917 data

Grow your own food, possible?  

Grow your own food, how far apart (urban life?)


\section{Example: How big could Tenochtitlan have been?}

1917 (A\&M) USDA pamphlet

Corn for US - area

If Tenoch was 100k people, how much land area?

\section{Example: Was the Irish Potato Famine a Natural Disaster?}

\section{Conclusion}


%\begin{acknowledgments}
\ack
The work was prompted in part by discussions with John Deming, Carl Ferkinhoff, and Sarah Taber.
%\end{acknowledgments}

%The command \appendix is used to signify the start of the appendices. Thereafter
%\section, \subsection, etc, will give headings appropriate for an appendix. To obtain
%a simple heading of ‘Appendix’ use the code \section*{Appendix}. If it contains
%numbered equations, figures or tables the command \appendix should precede it and
%\setcounter{section}{1} must follow it
\appendix
\section{Introductory Food Energy Questions}

Planning to save money, one college student decides to go to an all-you-can-eat buffet each day at 11am.  If he brings homework and stretches the meal out for a few hours he can get all $3000~kcals$ with only one meal bill.  Food is fuel for the human body. If his body burned all this food at once, how much warmer would he get? 
Useful information: the student has a mass of 80kg and is made mostly of water.  A Calorie heats 1 kg of water $1^{\circ}C$. 

Answer
\bea
3000kcals &=& 80kg\cdot1 \frac{kcal}{kg\cdot C^{\circ}}\cdot\Delta T\\
\Delta T &\approx& +37.5^{\circ}
\eea
Fat tissue serves a valuable purpose, brown fat, babies, songbirds

What power does the body give off in the more realistic case that the 3000kcal is burned over 24 hours? 
Useful information: $1 kcal \approx 4200J$ and $1 J/s=1W$.
\be
\frac{3000kcal}{24hours}\frac{4200J}{1kcal}\frac{1hour}{3600sec}\approx145W
\ee
Survival swimming, putting all the kids in one bed on a cold winter night.


\section*{References}
\begin{thebibliography}{99}

\bibitem{nature_cat}
Marey M
1894
Photographs of a Tumbling Cat. 
{\it Nature }
{\bf 51} 
80
%https://doi.org/10.1038/051080a0

\bibitem{Aztec_Cannibalism} for crop productivity

\bibitem{USDA_1917_yields}

\end{thebibliography}
\end{document}
