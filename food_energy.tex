% formatted for IOP
\documentclass[12pt]{iopart}

\pdfminorversion=4

\usepackage{float}
\usepackage{units}
\usepackage{graphicx}
\usepackage{hyperref}

\newcommand{\be}{\begin{equation}}
\newcommand{\ee}{\end{equation}}
\newcommand{\bea}{\begin{eqnarray}}
\newcommand{\eea}{\end{eqnarray}}
\begin{document}
\title[Food Energy]{Food Energy}
\author{N T Moore}
\footnote{Present address:
Department of Physics, Winona State University, Winona, MN 55987, USA}
\ead{nmoore@winona.edu}
\date{\today}
\begin{abstract}
Science and Social Policy classes are full of bespoke units and involve many different contexts.  Starting the class with a discussion of food energy is a nice way for everyone to start with the same context.  In addition, discussion of Food Energy can lead to interesting historical claims. 
\end{abstract}
\noindent{\it Keywords\/}: Calories, History, Energy, Potato Famine, Aztec Civilization
%\submitto{\PED}
\maketitle

\section{Introduction}
When the United States entered World War One one of the problems they faced was logistics.  How much food do you need to ship overseas to Europe to feed a million soldiers?  That early work in nutrition led to the 3000 Calorie diet many people remember from secondary Health Education class.  A bit about units you might remeber: $1 Calorie = 1 kilo-calorie$, and a dietician might build a 3000kcal diet for a 20 year old basketball player.  A \textit{calorie} is the amount of energy it takes (typically) to heat a gram of water by a degree Celsius.  There are about 4.2 Joules in a single calorie, and a Joule occurs all over introductory high-school physics.  If you need to buy a new electrical furnace, the sales brochure might advertise that it is capable of delivering 100,000 BTU's of heat each hour.  What's a BTU? Heat a pound of water by $1^{\circ}F$.  Of course Heat Pumps are far more efficient than simply burning methane or propane, but they consume kilo-watt-hours of electricity, not BTU's.  What's a kWh?  Run a 1000Watt toaster for an hour and you'll have pulled one off the grid, it will cost you about \$0.13 in Minnesota.  If you decide to put solar panels in your backyard, they will probably collect about $10\%$ of the 3.5kWh the  the sun delivers to each square meter of your lawn (in Minnesota) each day.  

There are a frustratingly large number of different units in an ``Energy'' class.  At Winona State, this 3 credit class fulfulls a ``Science and Social Policy'' general education requirement and is taken by students from across the university.   Lots of college majors don't require a math class beyond algebra or introductory statistics and the population is largely mathaverse. You could jokingly say that one of the main things students learn in the class is unit converstion, but it isn't far off.  Nearly every field finds energy a useful representation, and every profession has their own set of units and terminology that's most well suited.  Would a medical lab scientist talk about an acre-foot of urine needed test kidney function?  No, but someone in the central valley of California would certianly care about the acre-feet of water necessary to grow almonds!  Does a gas station price their gasoline in dollars per kWh? Given the growing electrification of cars, they might soon.

Everyone eats, maybe not 3000kcals per day, but at least something every day.  When I teach our energy class, I spend a few weeks talking about food energy before all other types.  While food production is not central to the current struggles with climate change and wars over oil, food is essential in a way that diesel and gasoline are not.  Vehicle fule makes modern life possible, but we could live, unpleasantly, without it.  We can't live without fats and protein though.  

\section{Food Energy}

Calories into body heat

Calories into power

Calories into climbing a hill, bar charts

Conversions: burn butter or oil or natural gas to heat your house

Conversions+bar charts, heat pump

\section{Example: How big could Tenochtitlan have been?}

1917 (A&M) USDA pamphlet

Corn for US - area

If Tenoch was 100k people, how much land area?

\section{Example: Was the Irish Potato Famine a Natural Disaster?}

\section{Conclusion}


%\begin{acknowledgments}
\ack
The work was prompted in part by discussions with John Deming, Carl Ferkinhoff, and Sarah Taber.
%\end{acknowledgments}



\section*{References}
\begin{thebibliography}{99}

\bibitem{nature_cat}
Marey M
1894
Photographs of a Tumbling Cat. 
{\it Nature }
{\bf 51} 
80
%https://doi.org/10.1038/051080a0

\bibitem{Aztec_Cannibalism} for crop productivity

\bibitem{USDA_1917_yields}

\end{thebibliography}
\end{document}
